% Copyright (C) 2019 Esote.
% Permission is granted to copy, distribute and/or modify this document
% under the terms of the GNU Free Documentation License, Version 1.3
% or any later version published by the Free Software Foundation;
% with no Invariant Sections, no Front-Cover Texts, and no Back-Cover Texts.
% A copy of the license is included in the section entitled "GNU
% Free Documentation License".

\documentclass{beamer}
\usepackage{hyperref}

\newcommand{\bhref}[2]{\href{#1}{\color{blue}{\underline{#2}}}}
\definecolor{darkgreen}{rgb}{0.0, 0.6, 0.0}

\mode<presentation> {
\usetheme{Madrid}
\setbeamertemplate{navigation symbols}{}
}

\title[PGP\hspace{1em}]{PGP}
\subtitle{Pretty Good Privacy and related things.}

\date{\today}

\begin{document}

\begin{frame}
	\titlepage
\end{frame}

\begin{frame}{Overview}
	\tableofcontents
\end{frame}

\section{History}

\begin{frame}{History}
	\begin{enumerate}
		\setlength{\itemsep}{1.5em}

		\item PGP began as a program written by Phil Zimmermann in 1991
		for symmetric encryption and decryption.

		\item Zimmermann was prosecuted by the US for the illegal export
		of munitions. Started PGP Inc., which was later bought out.

		\item Former PGP Inc. employees formed the PGP Corporation,
		which was later bought out.

		\item PGP Inc. proposed a standard to the IETF called OpenPGP.

		\item Two years later the FSF wrote GPG from OpenPGP.
	\end{enumerate}
\end{frame}

\section{Terminology}

\begin{frame}{Terminology}
	\textbf{PGP}:
	\begin{itemize}
		\item ``Pretty Good Privacy" is a program by Phil Zimmermann.

		\item Able to encrypt, decrypt, sign, and verify things using
		many kinds of cryptography.
	\end{itemize}

	\textbf{OpenPGP}:
	\begin{itemize}
		\item Standardization of PGP for digital interchange.

		\begin{itemize}
			\item Packets, ASCII armor, algorithm constants,
			fingerprints, keyrings, etc.
		\end{itemize}

		\item Came from the PGP program written by Zimmermann.

		\item ``Proposed standard" not a full standard, yet.
		\bhref{https://tools.ietf.org/html/rfc4880}{RFC 4880}.
	\end{itemize}

	\textbf{GPG}:
	\begin{itemize}
		\item ``GNU Privacy Guard" is an implementation of OpenPGP as
		part of the GNU Project.
	\end{itemize}
\end{frame}

\section{Using GPG}

\begin{frame}{Using GPG: keys}
	\bhref{https://www.gnupg.org/gph/en/manual/book1.html}{GPG online
	manual}.

	\small
	\def\arraystretch{2.8}

	\begin{tabular}{ll}
	Generating a key: & \texttt{\$ gpg2 --full-gen-key} \\

	Or, elliptic curves: & \texttt{\$ gpg2 --full-gen-key --expert} \\

	List public key: & \texttt{\$ gpg2 --list-keys} \\

	Export: & \texttt{\$ gpg2 --armor --export \{key\}} \\

	Receive key: & \texttt{\$ gpg2 --recv-keys \{key\} --keyserver \{URL\}}
	\\

	{\color{red}Send to keyserver:} & \texttt{\$ gpg2 --send-keys \{key\}
	--keyserver \{URL\}}
	\end{tabular}
\end{frame}

\begin{frame}[fragile]{Using GPG: encrypting}
	\setlength{\parskip}{1.5em}

	Use hidden recipients! \texttt{--hidden-recipient} rather than
	\texttt{--recipient}

	\small
	\def\arraystretch{3}

	\begin{tabular}{ll}
	Encrypt file: & \begin{minipage}{0em}
	\begin{verbatim}
	$ gpg2 --encrypt --hidden-recipient {key} \
       --output {output} {input}
       \end{verbatim}
       \end{minipage} \\

	Or, ASCII armor: & \texttt{\$ gpg2 --armor --encrypt ...}\\

	Decrypt file: & \texttt{\$ gpg2 --output \{output\} --decrypt \{input\}}
	\end{tabular}
\end{frame}

\begin{frame}{Using GPG: signing}
	\small
	\def\arraystretch{3}

	\begin{tabular}{ll}
	Sign and compress: & \texttt{\$ gpg2 --output \{output\} --sign
	\{input\}} \\

	Verify: & \texttt{\$ gpg2 --verify \{input\}} \\

	Verify and decompress: & \texttt{\$ gpg2 --output \{output\} --decrypt
	\{input\}} \\

	Clearsign: & \texttt{\$ gpg2 --output \{output\} --clearsign \{input\}}
	\\

	Detached signatures: & \texttt{\$ gpg2 --output \{output\} --detach-sig
	\{input\}} \\ \end{tabular}
\end{frame}

\section{Recent News}

\begin{frame}{Recent News: SKS Keyserver Poisoning}
	\setlength{\parskip}{1em}

	\bhref{https://gist.github.com/rjhansen/67ab921ffb4084c865b3618d6955275f}{Full
	write-up}.

	\begin{itemize}
		\item Keyservers are a registry of PGP keys.

		\item OpenPGP allows key signing: submitting a signature of
		someone's public key establishing your trust in the owner of
		that key. Sometimes done through ``key signing parties." This is
		essential to the web-of-trust model.

		\item SKS keyservers are a popular, high-performing, distributed
		pool of keyservers. SKS keyservers, and many other kinds of
		keyservers, do not stop random people from signing your key.
	\end{itemize}
\end{frame}

\begin{frame}{Recent News: SKS Keyserver Poisoning}
	\setlength{\parskip}{1em}

	The vulnerability is from spamming a key with signatures (order of
	150,000). This breaks GPG, but is isn't a vulnerability in the OpenPGP
	specification itself.

	Possible solutions:

	\begin{itemize}
		\item Modify the OpenPGP spec to require approval of key
		signatures before they attach themselves to a key.

		\item Modify keyservers to require some kind of verification or
		authentication for sending your key, or sending your key
		signatures.
	\end{itemize}
\end{frame}

\section{Cryptography}

\begin{frame}{Cryptography}
	\textbf{Symmetric cryptography}:
	\begin{itemize}
		\item Cryptographic algorithms that use the same key for
		encryption and decryption.

		\item Key should be kept secret.
	\end{itemize}

	\textbf{Asymmetric cryptography}:
	\begin{itemize}
		\item Cryptographic algorithms that use different keys for
		encryption and decryption.

		\item One key can be shared, one key should be kept secret.
	\end{itemize}

	\textbf{Digital signatures}:
	\begin{itemize}
		\item Use your private key to sign data, which can be verified
		through your public key.

		\item Provides integrity, authentication, and non-repudiation.
	\end{itemize}
\end{frame}

\begin{frame}{Cryptography}
	\textbf{RSA}:
	\begin{itemize}
		\item One such asymmetric cryptographic algorithm.

		\item Let $e, d, n$ be large positive numbers such that $\forall
		m$ where $0 \leqslant m \leqslant n$, $(m^e)^d \equiv m \mod n$.

		\item ... and
		\bhref{https://people.csail.mit.edu/rivest/Rsapaper.pdf}{lots}
		\bhref{https://patentimages.storage.googleapis.com/49/43/9c/b155bf231090f6/US4405829.pdf}{more}
		\bhref{https://github.com/libressl-portable/openbsd/tree/master/src/lib/libcrypto/rsa}{math}.
	\end{itemize}

	\textbf{DSA}:
	\begin{itemize}
		\item Like RSA, but faster for signing and decrypting, but
		slower for verifying and encrypting.
	\end{itemize}

	\textbf{ECC}:
	\begin{itemize}
		\item Elliptic-curve cryptography

		\item Smaller keys for equal strength crypto

		\item Be careful which curves you use
		\begin{itemize}
			\item {\color{darkgreen}Prefer: EdDSA / Ed25519 /
			Curve25519}

			\item {\color{red}Avoid: EcDSA (NSA backdoor)}
		\end{itemize}

		\item Weaker to quantum cryptography than RSA.
	\end{itemize}
\end{frame}

\section{Ramble}

\begin{frame}{Ramble}
	Ramble provides transparent and secure PGP-based messaging.

	\begin{itemize}
		\item Users introduce themselves to the server.

		\item Send messages as part of conversations.

		\item View messages.
	\end{itemize}

	All communication is done with your own PGP keys.

	\begin{itemize}
		\item Server requires key verification through nonce signatures
		before communicating.

		\item All messages are encrypted, the server cannot know their
		contents nor their recipients.

		\item \textbf{Even if a third party watches all connections they
		will gain no useful knowledge.}\footnote{Assuming you use hidden
		recipients for encrypted messages.}
	\end{itemize}

	\bhref{https://github.com/esote/ramble}{https://github.com/esote/ramble}
\end{frame}

\end{document}
